% Options for packages loaded elsewhere
\PassOptionsToPackage{unicode}{hyperref}
\PassOptionsToPackage{hyphens}{url}
\PassOptionsToPackage{dvipsnames,svgnames,x11names}{xcolor}
%
\documentclass[
  a4paper,
]{scrreport}

\usepackage{amsmath,amssymb}
\usepackage{lmodern}
\usepackage{iftex}
\ifPDFTeX
  \usepackage[T1]{fontenc}
  \usepackage[utf8]{inputenc}
  \usepackage{textcomp} % provide euro and other symbols
\else % if luatex or xetex
  \usepackage{unicode-math}
  \defaultfontfeatures{Scale=MatchLowercase}
  \defaultfontfeatures[\rmfamily]{Ligatures=TeX,Scale=1}
\fi
% Use upquote if available, for straight quotes in verbatim environments
\IfFileExists{upquote.sty}{\usepackage{upquote}}{}
\IfFileExists{microtype.sty}{% use microtype if available
  \usepackage[]{microtype}
  \UseMicrotypeSet[protrusion]{basicmath} % disable protrusion for tt fonts
}{}
\makeatletter
\@ifundefined{KOMAClassName}{% if non-KOMA class
  \IfFileExists{parskip.sty}{%
    \usepackage{parskip}
  }{% else
    \setlength{\parindent}{0pt}
    \setlength{\parskip}{6pt plus 2pt minus 1pt}}
}{% if KOMA class
  \KOMAoptions{parskip=half}}
\makeatother
\usepackage{xcolor}
\setlength{\emergencystretch}{3em} % prevent overfull lines
\setcounter{secnumdepth}{5}
% Make \paragraph and \subparagraph free-standing
\ifx\paragraph\undefined\else
  \let\oldparagraph\paragraph
  \renewcommand{\paragraph}[1]{\oldparagraph{#1}\mbox{}}
\fi
\ifx\subparagraph\undefined\else
  \let\oldsubparagraph\subparagraph
  \renewcommand{\subparagraph}[1]{\oldsubparagraph{#1}\mbox{}}
\fi

\usepackage{color}
\usepackage{fancyvrb}
\newcommand{\VerbBar}{|}
\newcommand{\VERB}{\Verb[commandchars=\\\{\}]}
\DefineVerbatimEnvironment{Highlighting}{Verbatim}{commandchars=\\\{\}}
% Add ',fontsize=\small' for more characters per line
\usepackage{framed}
\definecolor{shadecolor}{RGB}{241,243,245}
\newenvironment{Shaded}{\begin{snugshade}}{\end{snugshade}}
\newcommand{\AlertTok}[1]{\textcolor[rgb]{0.68,0.00,0.00}{#1}}
\newcommand{\AnnotationTok}[1]{\textcolor[rgb]{0.37,0.37,0.37}{#1}}
\newcommand{\AttributeTok}[1]{\textcolor[rgb]{0.40,0.45,0.13}{#1}}
\newcommand{\BaseNTok}[1]{\textcolor[rgb]{0.68,0.00,0.00}{#1}}
\newcommand{\BuiltInTok}[1]{\textcolor[rgb]{0.00,0.23,0.31}{#1}}
\newcommand{\CharTok}[1]{\textcolor[rgb]{0.13,0.47,0.30}{#1}}
\newcommand{\CommentTok}[1]{\textcolor[rgb]{0.37,0.37,0.37}{#1}}
\newcommand{\CommentVarTok}[1]{\textcolor[rgb]{0.37,0.37,0.37}{\textit{#1}}}
\newcommand{\ConstantTok}[1]{\textcolor[rgb]{0.56,0.35,0.01}{#1}}
\newcommand{\ControlFlowTok}[1]{\textcolor[rgb]{0.00,0.23,0.31}{#1}}
\newcommand{\DataTypeTok}[1]{\textcolor[rgb]{0.68,0.00,0.00}{#1}}
\newcommand{\DecValTok}[1]{\textcolor[rgb]{0.68,0.00,0.00}{#1}}
\newcommand{\DocumentationTok}[1]{\textcolor[rgb]{0.37,0.37,0.37}{\textit{#1}}}
\newcommand{\ErrorTok}[1]{\textcolor[rgb]{0.68,0.00,0.00}{#1}}
\newcommand{\ExtensionTok}[1]{\textcolor[rgb]{0.00,0.23,0.31}{#1}}
\newcommand{\FloatTok}[1]{\textcolor[rgb]{0.68,0.00,0.00}{#1}}
\newcommand{\FunctionTok}[1]{\textcolor[rgb]{0.28,0.35,0.67}{#1}}
\newcommand{\ImportTok}[1]{\textcolor[rgb]{0.00,0.46,0.62}{#1}}
\newcommand{\InformationTok}[1]{\textcolor[rgb]{0.37,0.37,0.37}{#1}}
\newcommand{\KeywordTok}[1]{\textcolor[rgb]{0.00,0.23,0.31}{#1}}
\newcommand{\NormalTok}[1]{\textcolor[rgb]{0.00,0.23,0.31}{#1}}
\newcommand{\OperatorTok}[1]{\textcolor[rgb]{0.37,0.37,0.37}{#1}}
\newcommand{\OtherTok}[1]{\textcolor[rgb]{0.00,0.23,0.31}{#1}}
\newcommand{\PreprocessorTok}[1]{\textcolor[rgb]{0.68,0.00,0.00}{#1}}
\newcommand{\RegionMarkerTok}[1]{\textcolor[rgb]{0.00,0.23,0.31}{#1}}
\newcommand{\SpecialCharTok}[1]{\textcolor[rgb]{0.37,0.37,0.37}{#1}}
\newcommand{\SpecialStringTok}[1]{\textcolor[rgb]{0.13,0.47,0.30}{#1}}
\newcommand{\StringTok}[1]{\textcolor[rgb]{0.13,0.47,0.30}{#1}}
\newcommand{\VariableTok}[1]{\textcolor[rgb]{0.07,0.07,0.07}{#1}}
\newcommand{\VerbatimStringTok}[1]{\textcolor[rgb]{0.13,0.47,0.30}{#1}}
\newcommand{\WarningTok}[1]{\textcolor[rgb]{0.37,0.37,0.37}{\textit{#1}}}

\providecommand{\tightlist}{%
  \setlength{\itemsep}{0pt}\setlength{\parskip}{0pt}}\usepackage{longtable,booktabs,array}
\usepackage{calc} % for calculating minipage widths
% Correct order of tables after \paragraph or \subparagraph
\usepackage{etoolbox}
\makeatletter
\patchcmd\longtable{\par}{\if@noskipsec\mbox{}\fi\par}{}{}
\makeatother
% Allow footnotes in longtable head/foot
\IfFileExists{footnotehyper.sty}{\usepackage{footnotehyper}}{\usepackage{footnote}}
\makesavenoteenv{longtable}
\usepackage{graphicx}
\makeatletter
\def\maxwidth{\ifdim\Gin@nat@width>\linewidth\linewidth\else\Gin@nat@width\fi}
\def\maxheight{\ifdim\Gin@nat@height>\textheight\textheight\else\Gin@nat@height\fi}
\makeatother
% Scale images if necessary, so that they will not overflow the page
% margins by default, and it is still possible to overwrite the defaults
% using explicit options in \includegraphics[width, height, ...]{}
\setkeys{Gin}{width=\maxwidth,height=\maxheight,keepaspectratio}
% Set default figure placement to htbp
\makeatletter
\def\fps@figure{htbp}
\makeatother

\usepackage{venndiagram}
\newcommand{\NN}{\mathbb{N}}
\newcommand{\ZZ}{\mathbb{Z}}
\newcommand{\QQ}{\mathbb{Q}}
\newcommand{\RR}{\mathbb{R}}
\newcommand{\CC}{\mathbb{C}}
\DeclareMathOperator{\operatorname{Int}}{Int}
\DeclareMathOperator{\operatorname{Ext}}{Ext}
\DeclareMathOperator{\operatorname{Fr}}{Fr}
\DeclareMathOperator{\Adh}{Adh}
\DeclareMathOperator{\Ac}{Ac}
\DeclareMathOperator{\sen}{sen}
\usepackage{booktabs}
\usepackage{longtable}
\usepackage{array}
\usepackage{multirow}
\usepackage{wrapfig}
\usepackage{float}
\usepackage{colortbl}
\usepackage{pdflscape}
\usepackage{tabu}
\usepackage{threeparttable}
\usepackage{threeparttablex}
\usepackage[normalem]{ulem}
\usepackage{makecell}
\usepackage{xcolor}
\makeatletter
\@ifpackageloaded{tcolorbox}{}{\usepackage[many]{tcolorbox}}
\@ifpackageloaded{fontawesome5}{}{\usepackage{fontawesome5}}
\definecolor{quarto-callout-color}{HTML}{909090}
\definecolor{quarto-callout-note-color}{HTML}{0758E5}
\definecolor{quarto-callout-important-color}{HTML}{CC1914}
\definecolor{quarto-callout-warning-color}{HTML}{EB9113}
\definecolor{quarto-callout-tip-color}{HTML}{00A047}
\definecolor{quarto-callout-caution-color}{HTML}{FC5300}
\definecolor{quarto-callout-color-frame}{HTML}{acacac}
\definecolor{quarto-callout-note-color-frame}{HTML}{4582ec}
\definecolor{quarto-callout-important-color-frame}{HTML}{d9534f}
\definecolor{quarto-callout-warning-color-frame}{HTML}{f0ad4e}
\definecolor{quarto-callout-tip-color-frame}{HTML}{02b875}
\definecolor{quarto-callout-caution-color-frame}{HTML}{fd7e14}
\makeatother
\makeatletter
\makeatother
\makeatletter
\@ifpackageloaded{bookmark}{}{\usepackage{bookmark}}
\makeatother
\makeatletter
\@ifpackageloaded{caption}{}{\usepackage{caption}}
\AtBeginDocument{%
\ifdefined\contentsname
  \renewcommand*\contentsname{Tabla de contenidos}
\else
  \newcommand\contentsname{Tabla de contenidos}
\fi
\ifdefined\listfigurename
  \renewcommand*\listfigurename{Listado de Figuras}
\else
  \newcommand\listfigurename{Listado de Figuras}
\fi
\ifdefined\listtablename
  \renewcommand*\listtablename{Listado de Tablas}
\else
  \newcommand\listtablename{Listado de Tablas}
\fi
\ifdefined\figurename
  \renewcommand*\figurename{Figura}
\else
  \newcommand\figurename{Figura}
\fi
\ifdefined\tablename
  \renewcommand*\tablename{Tabla}
\else
  \newcommand\tablename{Tabla}
\fi
}
\@ifpackageloaded{float}{}{\usepackage{float}}
\floatstyle{ruled}
\@ifundefined{c@chapter}{\newfloat{codelisting}{h}{lop}}{\newfloat{codelisting}{h}{lop}[chapter]}
\floatname{codelisting}{Listado}
\newcommand*\listoflistings{\listof{codelisting}{Listado de Listados}}
\usepackage{amsthm}
\theoremstyle{definition}
\newtheorem{exercise}{Ejercicio}[chapter]
\theoremstyle{remark}
\renewcommand*{\proofname}{Prueba}
\newtheorem*{remark}{Observación}
\newtheorem*{solution}{Solución}
\makeatother
\makeatletter
\@ifpackageloaded{caption}{}{\usepackage{caption}}
\@ifpackageloaded{subcaption}{}{\usepackage{subcaption}}
\makeatother
\makeatletter
\@ifpackageloaded{tcolorbox}{}{\usepackage[many]{tcolorbox}}
\makeatother
\makeatletter
\@ifundefined{shadecolor}{\definecolor{shadecolor}{rgb}{.97, .97, .97}}
\makeatother
\makeatletter
\makeatother
\ifLuaTeX
\usepackage[bidi=basic]{babel}
\else
\usepackage[bidi=default]{babel}
\fi
\babelprovide[main,import]{spanish}
% get rid of language-specific shorthands (see #6817):
\let\LanguageShortHands\languageshorthands
\def\languageshorthands#1{}
\ifLuaTeX
  \usepackage{selnolig}  % disable illegal ligatures
\fi
\IfFileExists{bookmark.sty}{\usepackage{bookmark}}{\usepackage{hyperref}}
\IfFileExists{xurl.sty}{\usepackage{xurl}}{} % add URL line breaks if available
\urlstyle{same} % disable monospaced font for URLs
\hypersetup{
  pdftitle={Prácticas de Estadística con R},
  pdfauthor={Alfredo Sánchez Alberca},
  pdflang={es},
  colorlinks=true,
  linkcolor={blue},
  filecolor={Maroon},
  citecolor={Blue},
  urlcolor={Blue},
  pdfcreator={LaTeX via pandoc}}

\title{Prácticas de Estadística con R}
\author{Alfredo Sánchez Alberca}
\date{6/1/22}

\begin{document}
\begin{titlepage}

%\AddToShipoutPicture*{\put(0,0){\includegraphics[scale=0.8]{img/background2}}} % Imagen de fondo, requiere el paquete eso-pic.
\begin{center}
\vspace*{5cm}

\Huge
{\textbf{\textsf{Prácticas de Estadística con R}}}

\vspace{0.5cm}
\LARGE
{\textbf{\textsf{}}}

\vspace{1.5cm}

\includegraphics[width=0.4\textwidth]{img/logos/sticker-estadistica-r.png}
\end{center}

\vfill

\begin{flushleft}
\begin{tabular}{ll}
\includegraphics[width=0.1\textwidth]{img/logos/aprendeconalf.png} & \parbox[b]{5cm}{\Large\textsf{Alfredo
Sánchez
Alberca}\\ \textsf{asalber@ceu.es} \\ \textsf{https://aprendeconalf.es}}
\end{tabular}
\end{flushleft}
\end{titlepage}\ifdefined\Shaded\renewenvironment{Shaded}{\begin{tcolorbox}[borderline west={3pt}{0pt}{shadecolor}, boxrule=0pt, sharp corners, breakable, interior hidden, frame hidden, enhanced]}{\end{tcolorbox}}\fi

\renewcommand*\contentsname{Tabla de contenidos}
{
\hypersetup{linkcolor=}
\setcounter{tocdepth}{2}
\tableofcontents
}
\bookmarksetup{startatroot}

\hypertarget{prefacio}{%
\chapter*{Prefacio}\label{prefacio}}
\addcontentsline{toc}{chapter}{Prefacio}

\markboth{Prefacio}{Prefacio}

¡Bienvenido a Prácticas de Estadística con R!

Este libro presenta una recopilación de prácticas de Estadística
Descriptiva e Inferencial con el lenguaje de programación
\href{https://www.r-project.org/}{R}, con problemas aplicados a las
Ciencias y las Ingenierías.

No es un libro para aprender a programar con R, ya que solo enseña el
uso del lenguaje y de algunos de sus paquetes para resolver problemas de
Estadística. Para quienes estén interesados en aprender a programar en
este lenguaje, os recomiendo leer este
\href{https://aprendeconalf.es/manual-r/}{manual de R}.

\hypertarget{capuxedtulos}{%
\section*{Capítulos}\label{capuxedtulos}}
\addcontentsline{toc}{section}{Capítulos}

\markright{Capítulos}

\hypertarget{toc}{}

\hypertarget{licencia}{%
\section*{Licencia}\label{licencia}}
\addcontentsline{toc}{section}{Licencia}

\markright{Licencia}

Esta obra está bajo una licencia Reconocimiento -- No comercial --
Compartir bajo la misma licencia 3.0 España de Creative Commons. Para
ver una copia de esta licencia, visite
\url{https://creativecommons.org/licenses/by-nc-sa/3.0/es/}.

Con esta licencia eres libre de:

\begin{itemize}
\tightlist
\item
  Copiar, distribuir y mostrar este trabajo.
\item
  Realizar modificaciones de este trabajo.
\end{itemize}

Bajo las siguientes condiciones:

\begin{itemize}
\item
  \textbf{Reconocimiento}. Debe reconocer los créditos de la obra de la
  manera especificada por el autor o el licenciador (pero no de una
  manera que sugiera que tiene su apoyo o apoyan el uso que hace de su
  obra).
\item
  \textbf{No comercial}. No puede utilizar esta obra para fines
  comerciales.
\item
  \textbf{Compartir bajo la misma licencia}. Si altera o transforma esta
  obra, o genera una obra derivada, sólo puede distribuir la obra
  generada bajo una licencia idéntica a ésta.
\end{itemize}

Al reutilizar o distribuir la obra, tiene que dejar bien claro los
términos de la licencia de esta obra.

Estas condiciones pueden no aplicarse si se obtiene el permiso del
titular de los derechos de autor.

Nada en esta licencia menoscaba o restringe los derechos morales del
autor.

\bookmarksetup{startatroot}

\hypertarget{introducciuxf3n-a-r}{%
\chapter{Introducción a R}\label{introducciuxf3n-a-r}}

La gran potencia de cómputo alcanzada por los ordenadores ha convertido
a los mismos en poderosas herramientas al servicio de todas aquellas
disciplinas que, como la Estadística, requieren manejar un gran volumen
de datos. Actualmente, prácticamente nadie se plantea hacer un estudio
estadístico serio sin la ayuda de un buen programa de análisis de datos.

\emph{R} es un potente lenguaje de programación que incluye multitud de
funciones para la representación y el análisis de datos. Fue
desarrollado por Robert Gentleman y Ross Ihaka en la Universidad de
Auckland en Nueva Zelanda, aunque actualmente es mantenido por una
enorme comunidad científica en todo el mundo.

\begin{figure}

{\centering \includegraphics[width=0.25\textwidth,height=\textheight]{./img/logos/Rlogo.png}

}

\caption{Logotipo de R}

\end{figure}

Las ventajas de R frente a otros programas habituales de análisis de
datos, como pueden ser SPSS, SAS o Matlab, son múltiples:

\begin{itemize}
\tightlist
\item
  Es software libre y por tanto gratuito. Puede descargarse desde la web
  \url{http://www.r-project.org/}.
\item
  Es multiplataforma. Existen versiones para Windows, Macintosh, Linux y
  otras plataformas.
\item
  Está avalado y en constante desarrollo por una amplia comunidad
  científica distribuida por todo el mundo que lo utiliza como estándar
  para el análisis de datos.
\item
  Cuenta con multitud de paquetes para todo tipo de análisis
  estadísticos y representaciones gráficas, desde los más habituales,
  hasta los más novedosos y sofisticados que no incluyen otros
  programas. Los paquetes están organizados y documentados en un
  \href{https://cran.r-project.org/}{repositorio CRAN} (Comprehensive R
  Archive Network) desde donde pueden descargarse libremente.
\item
  Es programable, lo que permite que el usuario pueda crear fácilmente
  sus propias funciones o paquetes para análisis de datos específicos.
  Existen multitud de libros, manuales y tutoriales libres que permiten
  su aprendizaje e ilustran el análisis estadístico de datos en
  distintas disciplinas científicas como las Matemáticas, la Física, la
  Biología, la Psicología, la Medicina, etc.
\end{itemize}

R puede descargarse desde el \href{https://www.r-project.org/}{sitio web
oficial de R} o desde el repositorio principal de paquetes de R
\href{https://cran.r-project.org/}{CRAN}. Basta con descargar el archivo
de instalación correspondiente al sistema operativo de nuestro ordenador
y realizar la instalación como cualquier otro programa.

El intérprete de R se arranca desde la terminal, aunque en Windows
incorpora su propia aplicación, pero es muy básica. En general, para
trabajos serios, conviene utilizar un entorno de desarrollo para R.

\hypertarget{entornos-de-desarrollo}{%
\section{Entornos de desarrollo}\label{entornos-de-desarrollo}}

Por defecto el entorno de trabajo de R es en línea de comandos, lo que
significa que los cálculos y los análisis se realizan mediante comandos
o instrucciones que el usuario teclea en una ventana de texto. No
obstante, existen distintas interfaces gráficas de usuario que facilitan
su uso, sobre todo para usuarios noveles. Algunas de ellas, como las que
se enumeran a continuación, son completos entornos de desarrollo que
facilitan la gestión de cualquier proyecto:

\begin{itemize}
\item
  \href{https://www.rstudio.com/}{RStudio}. Probablemente el entorno de
  desarrollo más extendido para programar con R ya que incorpora
  multitud de utilidades para facilitar la programación con R.
\item
  \href{https://rkward.kde.org}{RKWard}. Es otra otro de los entornos de
  desarrollo más completos que además incluye a posibilidad de añadir
  nuevos menús y cuadros de diálogo personalizados.
\item
  \href{https://code.visualstudio.com/}{Visual Studio Code}. Es un
  entorno de desarrollo de propósito general ampliamente extendido.
  Aunque no es un entorno de desarrollo específico para R, incluye una
  extensión con utilidades que facilitan mucho el desarrollo con R.
\end{itemize}

\bookmarksetup{startatroot}

\hypertarget{preprocesamiento-de-datos}{%
\chapter{Preprocesamiento de datos}\label{preprocesamiento-de-datos}}

Para la realización de esta práctica se requieren los paquetes
\texttt{readr} y
\href{https://aprendeconalf.es/manual-r/06-preprocesamiento.html\#el-paquete-dplyr}{\texttt{dplyr}}
de la colección de paquetes
\href{https://aprendeconalf.es/manual-r/06-preprocesamiento.html\#la-colecci\%C3\%B3n-de-paquetes-tidyverse}{\texttt{tidyverse}}.

\begin{Shaded}
\begin{Highlighting}[]
\FunctionTok{library}\NormalTok{(tidyverse) }
\CommentTok{\# Incluye los siguientes paquetes:}
\CommentTok{\# {-} readr: para la lectura de ficheros csv. }
\CommentTok{\# {-} dplyr: para el preprocesamiento y manipulación de datos.}
\end{Highlighting}
\end{Shaded}

\leavevmode\vadjust pre{\hypertarget{exr-preprocesamiento-1}{}}%
\begin{exercise}[]\label{exr-preprocesamiento-1}

La siguiente tabla contiene los ingresos y gastos de una empresa durante
el primer trimestre del año.

\begin{tabu} to \linewidth {>{\raggedright}X>{\raggedleft}X>{\raggedleft}X>{\raggedleft}X}
\hline
Mes & Ingresos & Gastos & Impuestos\\
\hline
Enero & 45000 & 33400 & 6450\\
\hline
Febrero & 41500 & 35400 & 6300\\
\hline
Marzo & 51200 & 35600 & 7100\\
\hline
\end{tabu}

\begin{enumerate}
\def\labelenumi{\alph{enumi}.}
\tightlist
\item
  Crear un data frame con los datos de la tabla.
\end{enumerate}

\begin{tcolorbox}[enhanced jigsaw, title=\textcolor{quarto-callout-tip-color}{\faLightbulb}\hspace{0.5em}{Solución}, coltitle=black, opacitybacktitle=0.6, rightrule=.15mm, colback=white, bottomtitle=1mm, breakable, leftrule=.75mm, opacityback=0, colbacktitle=quarto-callout-tip-color!10!white, left=2mm, colframe=quarto-callout-tip-color-frame, toptitle=1mm, titlerule=0mm, arc=.35mm, bottomrule=.15mm, toprule=.15mm]

\begin{Shaded}
\begin{Highlighting}[]
\NormalTok{df }\OtherTok{\textless{}{-}} \FunctionTok{data.frame}\NormalTok{(}
    \AttributeTok{Mes =} \FunctionTok{c}\NormalTok{(}\StringTok{"Enero"}\NormalTok{, }\StringTok{"Febrero"}\NormalTok{, }\StringTok{"Marzo"}\NormalTok{),}
    \AttributeTok{Ingresos =} \FunctionTok{c}\NormalTok{(}\DecValTok{45000}\NormalTok{, }\DecValTok{41500}\NormalTok{, }\DecValTok{51200}\NormalTok{),}
    \AttributeTok{Gastos =} \FunctionTok{c}\NormalTok{(}\DecValTok{33400}\NormalTok{, }\DecValTok{35400}\NormalTok{, }\DecValTok{35600}\NormalTok{)}
\NormalTok{    )}
\NormalTok{df }
\end{Highlighting}
\end{Shaded}

\begin{verbatim}
      Mes Ingresos Gastos
1   Enero    45000  33400
2 Febrero    41500  35400
3   Marzo    51200  35600
\end{verbatim}

\end{tcolorbox}

\begin{enumerate}
\def\labelenumi{\alph{enumi}.}
\setcounter{enumi}{1}
\tightlist
\item
  Añadir una nueva columna con los siguientes impuestos pagados.
\end{enumerate}

\begin{table}
\centering
\begin{tabular}{l|r}
\hline
Mes & Impuestos\\
\hline
Enero & 6450\\
\hline
Febrero & 6300\\
\hline
Marzo & 7100\\
\hline
\end{tabular}
\end{table}

\begin{tcolorbox}[enhanced jigsaw, title=\textcolor{quarto-callout-tip-color}{\faLightbulb}\hspace{0.5em}{Solución 1}, coltitle=black, opacitybacktitle=0.6, rightrule=.15mm, colback=white, bottomtitle=1mm, breakable, leftrule=.75mm, opacityback=0, colbacktitle=quarto-callout-tip-color!10!white, left=2mm, colframe=quarto-callout-tip-color-frame, toptitle=1mm, titlerule=0mm, arc=.35mm, bottomrule=.15mm, toprule=.15mm]

Con las funciones básicas de R.

\begin{Shaded}
\begin{Highlighting}[]
\NormalTok{df}\SpecialCharTok{$}\NormalTok{Impuestos }\OtherTok{\textless{}{-}} \FunctionTok{c}\NormalTok{(}\DecValTok{6450}\NormalTok{, }\DecValTok{6300}\NormalTok{, }\DecValTok{7100}\NormalTok{)}
\NormalTok{df}
\end{Highlighting}
\end{Shaded}

\begin{verbatim}
      Mes Ingresos Gastos Impuestos
1   Enero    45000  33400      6450
2 Febrero    41500  35400      6300
3   Marzo    51200  35600      7100
\end{verbatim}

\end{tcolorbox}

\begin{tcolorbox}[enhanced jigsaw, title=\textcolor{quarto-callout-tip-color}{\faLightbulb}\hspace{0.5em}{Solución 2}, coltitle=black, opacitybacktitle=0.6, rightrule=.15mm, colback=white, bottomtitle=1mm, breakable, leftrule=.75mm, opacityback=0, colbacktitle=quarto-callout-tip-color!10!white, left=2mm, colframe=quarto-callout-tip-color-frame, toptitle=1mm, titlerule=0mm, arc=.35mm, bottomrule=.15mm, toprule=.15mm]

Con las funciones del paquete \texttt{dplyr}.

\begin{Shaded}
\begin{Highlighting}[]
\NormalTok{df }\OtherTok{\textless{}{-}}\NormalTok{ df }\SpecialCharTok{\%\textgreater{}\%}
    \FunctionTok{mutate}\NormalTok{(}\AttributeTok{Impuestos =} \FunctionTok{c}\NormalTok{(}\DecValTok{6450}\NormalTok{, }\DecValTok{6300}\NormalTok{, }\DecValTok{7100}\NormalTok{))}
\NormalTok{df}
\end{Highlighting}
\end{Shaded}

\begin{verbatim}
      Mes Ingresos Gastos Impuestos
1   Enero    45000  33400      6450
2 Febrero    41500  35400      6300
3   Marzo    51200  35600      7100
\end{verbatim}

\end{tcolorbox}

\begin{enumerate}
\def\labelenumi{\alph{enumi}.}
\setcounter{enumi}{2}
\tightlist
\item
  Añadir una nueva fila con los siguientes datos de Abril.
\end{enumerate}

\begin{tabu} to \linewidth {>{\raggedright}X>{\raggedleft}X>{\raggedleft}X>{\raggedleft}X}
\hline
Mes & Ingresos & Gastos & Impuestos\\
\hline
Abril & 49700 & 36300 & 6850\\
\hline
\end{tabu}

\begin{tcolorbox}[enhanced jigsaw, title=\textcolor{quarto-callout-tip-color}{\faLightbulb}\hspace{0.5em}{Solución 1}, coltitle=black, opacitybacktitle=0.6, rightrule=.15mm, colback=white, bottomtitle=1mm, breakable, leftrule=.75mm, opacityback=0, colbacktitle=quarto-callout-tip-color!10!white, left=2mm, colframe=quarto-callout-tip-color-frame, toptitle=1mm, titlerule=0mm, arc=.35mm, bottomrule=.15mm, toprule=.15mm]

Con las funciones básicas de R.

\begin{Shaded}
\begin{Highlighting}[]
\NormalTok{df }\OtherTok{\textless{}{-}} \FunctionTok{rbind}\NormalTok{(df, }\FunctionTok{list}\NormalTok{(}\AttributeTok{Mes =} \StringTok{"Abril"}\NormalTok{, }\AttributeTok{Ingresos =} \DecValTok{49700}\NormalTok{, }\AttributeTok{Gastos =} \DecValTok{36300}\NormalTok{, }\AttributeTok{Impuestos =} \DecValTok{6850}\NormalTok{))}
\NormalTok{df}
\end{Highlighting}
\end{Shaded}

\begin{verbatim}
      Mes Ingresos Gastos Impuestos
1   Enero    45000  33400      6450
2 Febrero    41500  35400      6300
3   Marzo    51200  35600      7100
4   Abril    49700  36300      6850
\end{verbatim}

\end{tcolorbox}

\begin{tcolorbox}[enhanced jigsaw, title=\textcolor{quarto-callout-tip-color}{\faLightbulb}\hspace{0.5em}{Solución 2}, coltitle=black, opacitybacktitle=0.6, rightrule=.15mm, colback=white, bottomtitle=1mm, breakable, leftrule=.75mm, opacityback=0, colbacktitle=quarto-callout-tip-color!10!white, left=2mm, colframe=quarto-callout-tip-color-frame, toptitle=1mm, titlerule=0mm, arc=.35mm, bottomrule=.15mm, toprule=.15mm]

Con las funciones del paquete \texttt{dplyr}.

\begin{Shaded}
\begin{Highlighting}[]
\NormalTok{df }\OtherTok{\textless{}{-}}\NormalTok{ df }\SpecialCharTok{\%\textgreater{}\%}
    \FunctionTok{add\_row}\NormalTok{(}\AttributeTok{Mes =} \StringTok{"Abril"}\NormalTok{, }\AttributeTok{Ingresos =} \DecValTok{49700}\NormalTok{, }\AttributeTok{Gastos =} \DecValTok{36300}\NormalTok{, }\AttributeTok{Impuestos =} \DecValTok{6850}\NormalTok{)}
\NormalTok{df}
\end{Highlighting}
\end{Shaded}

\begin{verbatim}
      Mes Ingresos Gastos Impuestos
1   Enero    45000  33400      6450
2 Febrero    41500  35400      6300
3   Marzo    51200  35600      7100
4   Abril    49700  36300      6850
\end{verbatim}

\end{tcolorbox}

\begin{enumerate}
\def\labelenumi{\alph{enumi}.}
\setcounter{enumi}{3}
\tightlist
\item
  Cambiar los ingresos de Marzo por 50400.
\end{enumerate}

\begin{tcolorbox}[enhanced jigsaw, title=\textcolor{quarto-callout-tip-color}{\faLightbulb}\hspace{0.5em}{Solución}, coltitle=black, opacitybacktitle=0.6, rightrule=.15mm, colback=white, bottomtitle=1mm, breakable, leftrule=.75mm, opacityback=0, colbacktitle=quarto-callout-tip-color!10!white, left=2mm, colframe=quarto-callout-tip-color-frame, toptitle=1mm, titlerule=0mm, arc=.35mm, bottomrule=.15mm, toprule=.15mm]

\begin{Shaded}
\begin{Highlighting}[]
\NormalTok{df[}\DecValTok{3}\NormalTok{, }\StringTok{"Ingresos"}\NormalTok{] }\OtherTok{\textless{}{-}} \DecValTok{50400}
\NormalTok{df}
\end{Highlighting}
\end{Shaded}

\begin{verbatim}
      Mes Ingresos Gastos Impuestos
1   Enero    45000  33400      6450
2 Febrero    41500  35400      6300
3   Marzo    50400  35600      7100
4   Abril    49700  36300      6850
\end{verbatim}

\end{tcolorbox}

\begin{enumerate}
\def\labelenumi{\alph{enumi}.}
\setcounter{enumi}{4}
\tightlist
\item
  Crear una nueva columna con los beneficios de cada mes (ingresos -
  gastos - impuestos).
\end{enumerate}

\begin{tcolorbox}[enhanced jigsaw, title=\textcolor{quarto-callout-tip-color}{\faLightbulb}\hspace{0.5em}{Solución 1}, coltitle=black, opacitybacktitle=0.6, rightrule=.15mm, colback=white, bottomtitle=1mm, breakable, leftrule=.75mm, opacityback=0, colbacktitle=quarto-callout-tip-color!10!white, left=2mm, colframe=quarto-callout-tip-color-frame, toptitle=1mm, titlerule=0mm, arc=.35mm, bottomrule=.15mm, toprule=.15mm]

Con las funciones básicas de R.

\begin{Shaded}
\begin{Highlighting}[]
\NormalTok{df}\SpecialCharTok{$}\NormalTok{Beneficios }\OtherTok{\textless{}{-}}\NormalTok{ df}\SpecialCharTok{$}\NormalTok{Ingresos }\SpecialCharTok{{-}}\NormalTok{ df}\SpecialCharTok{$}\NormalTok{Gastos }\SpecialCharTok{{-}}\NormalTok{ df}\SpecialCharTok{$}\NormalTok{Impuestos}
\NormalTok{df}
\end{Highlighting}
\end{Shaded}

\begin{verbatim}
      Mes Ingresos Gastos Impuestos Beneficios
1   Enero    45000  33400      6450       5150
2 Febrero    41500  35400      6300       -200
3   Marzo    50400  35600      7100       7700
4   Abril    49700  36300      6850       6550
\end{verbatim}

\end{tcolorbox}

\begin{tcolorbox}[enhanced jigsaw, title=\textcolor{quarto-callout-tip-color}{\faLightbulb}\hspace{0.5em}{Solución 2}, coltitle=black, opacitybacktitle=0.6, rightrule=.15mm, colback=white, bottomtitle=1mm, breakable, leftrule=.75mm, opacityback=0, colbacktitle=quarto-callout-tip-color!10!white, left=2mm, colframe=quarto-callout-tip-color-frame, toptitle=1mm, titlerule=0mm, arc=.35mm, bottomrule=.15mm, toprule=.15mm]

Con las funciones del paquete \texttt{dplyr}.

\begin{Shaded}
\begin{Highlighting}[]
\NormalTok{df }\OtherTok{\textless{}{-}}\NormalTok{ df }\SpecialCharTok{\%\textgreater{}\%}
    \FunctionTok{mutate}\NormalTok{(}\AttributeTok{Beneficios =}\NormalTok{ Ingresos }\SpecialCharTok{{-}}\NormalTok{ Gastos }\SpecialCharTok{{-}}\NormalTok{ Impuestos)}
\NormalTok{df}
\end{Highlighting}
\end{Shaded}

\begin{verbatim}
      Mes Ingresos Gastos Impuestos Beneficios
1   Enero    45000  33400      6450       5150
2 Febrero    41500  35400      6300       -200
3   Marzo    50400  35600      7100       7700
4   Abril    49700  36300      6850       6550
\end{verbatim}

\end{tcolorbox}

\begin{enumerate}
\def\labelenumi{\alph{enumi}.}
\setcounter{enumi}{5}
\tightlist
\item
  Crear una nueva columna con el factor \texttt{Balance} con dos
  posibles categorías: \texttt{positivo} si ha habido beneficios y
  \texttt{negativo} si ha habido pérdidas.
\end{enumerate}

\begin{tcolorbox}[enhanced jigsaw, title=\textcolor{quarto-callout-tip-color}{\faLightbulb}\hspace{0.5em}{Solución 1}, coltitle=black, opacitybacktitle=0.6, rightrule=.15mm, colback=white, bottomtitle=1mm, breakable, leftrule=.75mm, opacityback=0, colbacktitle=quarto-callout-tip-color!10!white, left=2mm, colframe=quarto-callout-tip-color-frame, toptitle=1mm, titlerule=0mm, arc=.35mm, bottomrule=.15mm, toprule=.15mm]

Con las funciones básicas de R.

\begin{Shaded}
\begin{Highlighting}[]
\NormalTok{df}\SpecialCharTok{$}\NormalTok{Balance }\OtherTok{\textless{}{-}} \FunctionTok{cut}\NormalTok{(df}\SpecialCharTok{$}\NormalTok{Beneficios, }\AttributeTok{breaks =} \FunctionTok{c}\NormalTok{(}\SpecialCharTok{{-}}\ConstantTok{Inf}\NormalTok{, }\DecValTok{0}\NormalTok{, }\ConstantTok{Inf}\NormalTok{), }\AttributeTok{labels =} \FunctionTok{c}\NormalTok{(}\StringTok{"negativo"}\NormalTok{, }\StringTok{"positivo"}\NormalTok{))}
\NormalTok{df}
\end{Highlighting}
\end{Shaded}

\begin{verbatim}
      Mes Ingresos Gastos Impuestos Beneficios  Balance
1   Enero    45000  33400      6450       5150 positivo
2 Febrero    41500  35400      6300       -200 negativo
3   Marzo    50400  35600      7100       7700 positivo
4   Abril    49700  36300      6850       6550 positivo
\end{verbatim}

\end{tcolorbox}

\begin{tcolorbox}[enhanced jigsaw, title=\textcolor{quarto-callout-tip-color}{\faLightbulb}\hspace{0.5em}{Solución 2}, coltitle=black, opacitybacktitle=0.6, rightrule=.15mm, colback=white, bottomtitle=1mm, breakable, leftrule=.75mm, opacityback=0, colbacktitle=quarto-callout-tip-color!10!white, left=2mm, colframe=quarto-callout-tip-color-frame, toptitle=1mm, titlerule=0mm, arc=.35mm, bottomrule=.15mm, toprule=.15mm]

Con las funciones del paquete \texttt{dplyr}.

\begin{Shaded}
\begin{Highlighting}[]
\NormalTok{df }\OtherTok{\textless{}{-}}\NormalTok{ df }\SpecialCharTok{\%\textgreater{}\%}
    \FunctionTok{mutate}\NormalTok{(}\AttributeTok{Balance =} \FunctionTok{cut}\NormalTok{(Beneficios, }\AttributeTok{breaks =} \FunctionTok{c}\NormalTok{(}\SpecialCharTok{{-}}\ConstantTok{Inf}\NormalTok{, }\DecValTok{0}\NormalTok{, }\ConstantTok{Inf}\NormalTok{), }\AttributeTok{labels =} \FunctionTok{c}\NormalTok{(}\StringTok{"negativo"}\NormalTok{, }\StringTok{"positivo"}\NormalTok{)))}
\NormalTok{df}
\end{Highlighting}
\end{Shaded}

\begin{verbatim}
      Mes Ingresos Gastos Impuestos Beneficios  Balance
1   Enero    45000  33400      6450       5150 positivo
2 Febrero    41500  35400      6300       -200 negativo
3   Marzo    50400  35600      7100       7700 positivo
4   Abril    49700  36300      6850       6550 positivo
\end{verbatim}

\end{tcolorbox}

\begin{enumerate}
\def\labelenumi{\alph{enumi}.}
\setcounter{enumi}{6}
\tightlist
\item
  Filtrar el conjunto de datos para quedarse con los nombres de los
  meses y los beneficios de los meses con balance positivo.
\end{enumerate}

\begin{tcolorbox}[enhanced jigsaw, title=\textcolor{quarto-callout-tip-color}{\faLightbulb}\hspace{0.5em}{Solución 1}, coltitle=black, opacitybacktitle=0.6, rightrule=.15mm, colback=white, bottomtitle=1mm, breakable, leftrule=.75mm, opacityback=0, colbacktitle=quarto-callout-tip-color!10!white, left=2mm, colframe=quarto-callout-tip-color-frame, toptitle=1mm, titlerule=0mm, arc=.35mm, bottomrule=.15mm, toprule=.15mm]

Con las funciones básicas de R.

\begin{Shaded}
\begin{Highlighting}[]
\NormalTok{df[df}\SpecialCharTok{$}\NormalTok{Balance }\SpecialCharTok{==} \StringTok{"positivo"}\NormalTok{, }\FunctionTok{c}\NormalTok{(}\StringTok{"Mes"}\NormalTok{, }\StringTok{"Beneficios"}\NormalTok{)]}
\end{Highlighting}
\end{Shaded}

\begin{verbatim}
    Mes Beneficios
1 Enero       5150
3 Marzo       7700
4 Abril       6550
\end{verbatim}

\end{tcolorbox}

\begin{tcolorbox}[enhanced jigsaw, title=\textcolor{quarto-callout-tip-color}{\faLightbulb}\hspace{0.5em}{Solución 2}, coltitle=black, opacitybacktitle=0.6, rightrule=.15mm, colback=white, bottomtitle=1mm, breakable, leftrule=.75mm, opacityback=0, colbacktitle=quarto-callout-tip-color!10!white, left=2mm, colframe=quarto-callout-tip-color-frame, toptitle=1mm, titlerule=0mm, arc=.35mm, bottomrule=.15mm, toprule=.15mm]

Con las funciones del paquete \texttt{dplyr}.

\begin{Shaded}
\begin{Highlighting}[]
\NormalTok{df }\SpecialCharTok{\%\textgreater{}\%}
    \FunctionTok{filter}\NormalTok{(Balance }\SpecialCharTok{==} \StringTok{"positivo"}\NormalTok{) }\SpecialCharTok{\%\textgreater{}\%} 
    \FunctionTok{select}\NormalTok{(Mes, Beneficios)}
\end{Highlighting}
\end{Shaded}

\begin{verbatim}
    Mes Beneficios
1 Enero       5150
2 Marzo       7700
3 Abril       6550
\end{verbatim}

\end{tcolorbox}

\end{exercise}

\leavevmode\vadjust pre{\hypertarget{exr-preprocesamiento-2}{}}%
\begin{exercise}[]\label{exr-preprocesamiento-2}

El fichero
\href{\textquotesingle{}https://raw.githubusercontent.com/asalber/manual-r/master/datos/colesterol.csv\textquotesingle{}}{\texttt{colesterol.csv}}
contiene información de una muestra de pacientes donde se han medido la
edad, el sexo, el peso, la altura y el nivel de colesterol, además de su
nombre.

\begin{enumerate}
\def\labelenumi{\alph{enumi}.}
\tightlist
\item
  Crear un data frame con los datos de todos los pacientes del estudio a
  partir del fichero
  \href{(\textquotesingle{}https://raw.githubusercontent.com/asalber/manual-r/master/datos/colesterol.csv\textquotesingle{})}{\texttt{colesterol.csv}}.
\end{enumerate}

\begin{tcolorbox}[enhanced jigsaw, title=\textcolor{quarto-callout-tip-color}{\faLightbulb}\hspace{0.5em}{Solución 1}, coltitle=black, opacitybacktitle=0.6, rightrule=.15mm, colback=white, bottomtitle=1mm, breakable, leftrule=.75mm, opacityback=0, colbacktitle=quarto-callout-tip-color!10!white, left=2mm, colframe=quarto-callout-tip-color-frame, toptitle=1mm, titlerule=0mm, arc=.35mm, bottomrule=.15mm, toprule=.15mm]

Con las funciones básicas de R.

\begin{Shaded}
\begin{Highlighting}[]
\NormalTok{df }\OtherTok{\textless{}{-}} \FunctionTok{read.csv}\NormalTok{(}\StringTok{\textquotesingle{}https://raw.githubusercontent.com/asalber/manual{-}r/master/datos/colesterol.csv\textquotesingle{}}\NormalTok{)}
\NormalTok{df}
\end{Highlighting}
\end{Shaded}

\begin{verbatim}
                            nombre edad sexo peso altura colesterol
1     José Luis Martínez Izquierdo   18    H   85   1.79        182
2                   Rosa Díaz Díaz   32    M   65   1.73        232
3            Javier García Sánchez   24    H   NA   1.81        191
4              Carmen López Pinzón   35    M   65   1.70        200
5             Marisa López Collado   46    M   51   1.58        148
6                Antonio Ruiz Cruz   68    H   66   1.74        249
7          Antonio Fernández Ocaña   51    H   62   1.72        276
8            Pilar Martín González   22    M   60   1.66         NA
9             Pedro Gálvez Tenorio   35    H   90   1.94        241
10         Santiago Reillo Manzano   46    H   75   1.85        280
11           Macarena Álvarez Luna   53    M   55   1.62        262
12      José María de la Guía Sanz   58    H   78   1.87        198
13 Miguel Angel Cuadrado Gutiérrez   27    H  109   1.98        210
14           Carolina Rubio Moreno   20    M   61   1.77        194
\end{verbatim}

\end{tcolorbox}

\begin{tcolorbox}[enhanced jigsaw, title=\textcolor{quarto-callout-tip-color}{\faLightbulb}\hspace{0.5em}{Solución 2}, coltitle=black, opacitybacktitle=0.6, rightrule=.15mm, colback=white, bottomtitle=1mm, breakable, leftrule=.75mm, opacityback=0, colbacktitle=quarto-callout-tip-color!10!white, left=2mm, colframe=quarto-callout-tip-color-frame, toptitle=1mm, titlerule=0mm, arc=.35mm, bottomrule=.15mm, toprule=.15mm]

Con las funciones del paquete \texttt{readr}.

\begin{Shaded}
\begin{Highlighting}[]
\NormalTok{df }\OtherTok{\textless{}{-}} \FunctionTok{read\_csv}\NormalTok{(}\StringTok{\textquotesingle{}https://raw.githubusercontent.com/asalber/manual{-}r/master/datos/colesterol.csv\textquotesingle{}}\NormalTok{)}
\NormalTok{df}
\end{Highlighting}
\end{Shaded}

\begin{verbatim}
# A tibble: 14 x 6
   nombre                           edad sexo   peso altura colesterol
   <chr>                           <dbl> <chr> <dbl>  <dbl>      <dbl>
 1 José Luis Martínez Izquierdo       18 H        85   1.79        182
 2 Rosa Díaz Díaz                     32 M        65   1.73        232
 3 Javier García Sánchez              24 H        NA   1.81        191
 4 Carmen López Pinzón                35 M        65   1.7         200
 5 Marisa López Collado               46 M        51   1.58        148
 6 Antonio Ruiz Cruz                  68 H        66   1.74        249
 7 Antonio Fernández Ocaña            51 H        62   1.72        276
 8 Pilar Martín González              22 M        60   1.66         NA
 9 Pedro Gálvez Tenorio               35 H        90   1.94        241
10 Santiago Reillo Manzano            46 H        75   1.85        280
11 Macarena Álvarez Luna              53 M        55   1.62        262
12 José María de la Guía Sanz         58 H        78   1.87        198
13 Miguel Angel Cuadrado Gutiérrez    27 H       109   1.98        210
14 Carolina Rubio Moreno              20 M        61   1.77        194
\end{verbatim}

\end{tcolorbox}

\begin{enumerate}
\def\labelenumi{\alph{enumi}.}
\setcounter{enumi}{1}
\tightlist
\item
  Crear una nueva columna con el índice de masa corporal, usando la
  siguiente fórmula
\end{enumerate}

\[
\mbox{IMC} = \frac{\mbox{Peso (kg)}}{\mbox{Altura (cm)}^2}
\]

\begin{tcolorbox}[enhanced jigsaw, title=\textcolor{quarto-callout-tip-color}{\faLightbulb}\hspace{0.5em}{Solución}, coltitle=black, opacitybacktitle=0.6, rightrule=.15mm, colback=white, bottomtitle=1mm, breakable, leftrule=.75mm, opacityback=0, colbacktitle=quarto-callout-tip-color!10!white, left=2mm, colframe=quarto-callout-tip-color-frame, toptitle=1mm, titlerule=0mm, arc=.35mm, bottomrule=.15mm, toprule=.15mm]

\begin{Shaded}
\begin{Highlighting}[]
\NormalTok{df }\OtherTok{\textless{}{-}}\NormalTok{ df }\SpecialCharTok{\%\textgreater{}\%}
    \FunctionTok{mutate}\NormalTok{(}\AttributeTok{imc =} \FunctionTok{round}\NormalTok{(peso}\SpecialCharTok{/}\NormalTok{altura}\SpecialCharTok{\^{}}\DecValTok{2}\NormalTok{))}
\NormalTok{df}
\end{Highlighting}
\end{Shaded}

\begin{verbatim}
# A tibble: 14 x 7
   nombre                           edad sexo   peso altura colesterol   imc
   <chr>                           <dbl> <chr> <dbl>  <dbl>      <dbl> <dbl>
 1 José Luis Martínez Izquierdo       18 H        85   1.79        182    27
 2 Rosa Díaz Díaz                     32 M        65   1.73        232    22
 3 Javier García Sánchez              24 H        NA   1.81        191    NA
 4 Carmen López Pinzón                35 M        65   1.7         200    22
 5 Marisa López Collado               46 M        51   1.58        148    20
 6 Antonio Ruiz Cruz                  68 H        66   1.74        249    22
 7 Antonio Fernández Ocaña            51 H        62   1.72        276    21
 8 Pilar Martín González              22 M        60   1.66         NA    22
 9 Pedro Gálvez Tenorio               35 H        90   1.94        241    24
10 Santiago Reillo Manzano            46 H        75   1.85        280    22
11 Macarena Álvarez Luna              53 M        55   1.62        262    21
12 José María de la Guía Sanz         58 H        78   1.87        198    22
13 Miguel Angel Cuadrado Gutiérrez    27 H       109   1.98        210    28
14 Carolina Rubio Moreno              20 M        61   1.77        194    19
\end{verbatim}

\end{tcolorbox}

\begin{enumerate}
\def\labelenumi{\alph{enumi}.}
\setcounter{enumi}{2}
\tightlist
\item
  Crear una nueva columna con la variable \texttt{obesidad}
  recodificando la columna \texttt{imc} en las siguientes categorías.
\end{enumerate}

\begin{longtable}[]{@{}ll@{}}
\toprule()
Rango IMC & Categoría \\
\midrule()
\endhead
Menor de 18.5 & Bajo peso \\
De 18.5 a 24.5 & Saludable \\
De 24.5 a 30 & Sobrepeso \\
Mayor de 30 & Obeso \\
\bottomrule()
\end{longtable}

\begin{tcolorbox}[enhanced jigsaw, title=\textcolor{quarto-callout-tip-color}{\faLightbulb}\hspace{0.5em}{Solución}, coltitle=black, opacitybacktitle=0.6, rightrule=.15mm, colback=white, bottomtitle=1mm, breakable, leftrule=.75mm, opacityback=0, colbacktitle=quarto-callout-tip-color!10!white, left=2mm, colframe=quarto-callout-tip-color-frame, toptitle=1mm, titlerule=0mm, arc=.35mm, bottomrule=.15mm, toprule=.15mm]

\begin{Shaded}
\begin{Highlighting}[]
\NormalTok{df }\OtherTok{\textless{}{-}}\NormalTok{ df }\SpecialCharTok{\%\textgreater{}\%}
    \FunctionTok{mutate}\NormalTok{(}\AttributeTok{Obesidad =} \FunctionTok{cut}\NormalTok{(imc, }\AttributeTok{breaks =} \FunctionTok{c}\NormalTok{(}\DecValTok{0}\NormalTok{, }\FloatTok{18.5}\NormalTok{, }\FloatTok{24.5}\NormalTok{, }\DecValTok{30}\NormalTok{, }\ConstantTok{Inf}\NormalTok{), }\AttributeTok{labels =} \FunctionTok{c}\NormalTok{(}\StringTok{"Bajo peso"}\NormalTok{, }\StringTok{"Saludable"}\NormalTok{, }\StringTok{"Sobrepeso"}\NormalTok{, }\StringTok{"Obeso"}\NormalTok{)))}
\NormalTok{df}
\end{Highlighting}
\end{Shaded}

\begin{verbatim}
# A tibble: 14 x 8
   nombre                      edad sexo   peso altura colesterol   imc Obesidad
   <chr>                      <dbl> <chr> <dbl>  <dbl>      <dbl> <dbl> <fct>   
 1 José Luis Martínez Izquie~    18 H        85   1.79        182    27 Sobrepe~
 2 Rosa Díaz Díaz                32 M        65   1.73        232    22 Saludab~
 3 Javier García Sánchez         24 H        NA   1.81        191    NA <NA>    
 4 Carmen López Pinzón           35 M        65   1.7         200    22 Saludab~
 5 Marisa López Collado          46 M        51   1.58        148    20 Saludab~
 6 Antonio Ruiz Cruz             68 H        66   1.74        249    22 Saludab~
 7 Antonio Fernández Ocaña       51 H        62   1.72        276    21 Saludab~
 8 Pilar Martín González         22 M        60   1.66         NA    22 Saludab~
 9 Pedro Gálvez Tenorio          35 H        90   1.94        241    24 Saludab~
10 Santiago Reillo Manzano       46 H        75   1.85        280    22 Saludab~
11 Macarena Álvarez Luna         53 M        55   1.62        262    21 Saludab~
12 José María de la Guía Sanz    58 H        78   1.87        198    22 Saludab~
13 Miguel Angel Cuadrado Gut~    27 H       109   1.98        210    28 Sobrepe~
14 Carolina Rubio Moreno         20 M        61   1.77        194    19 Saludab~
\end{verbatim}

\end{tcolorbox}

\begin{enumerate}
\def\labelenumi{\alph{enumi}.}
\setcounter{enumi}{3}
\tightlist
\item
  Seleccionar las columnas \texttt{nombre}, \texttt{sexo} y
  \texttt{edad}.
\end{enumerate}

\begin{tcolorbox}[enhanced jigsaw, title=\textcolor{quarto-callout-tip-color}{\faLightbulb}\hspace{0.5em}{Solución}, coltitle=black, opacitybacktitle=0.6, rightrule=.15mm, colback=white, bottomtitle=1mm, breakable, leftrule=.75mm, opacityback=0, colbacktitle=quarto-callout-tip-color!10!white, left=2mm, colframe=quarto-callout-tip-color-frame, toptitle=1mm, titlerule=0mm, arc=.35mm, bottomrule=.15mm, toprule=.15mm]

\begin{Shaded}
\begin{Highlighting}[]
\NormalTok{df }\SpecialCharTok{\%\textgreater{}\%}
    \FunctionTok{select}\NormalTok{(nombre, sexo, edad)}
\end{Highlighting}
\end{Shaded}

\begin{verbatim}
# A tibble: 14 x 3
   nombre                          sexo   edad
   <chr>                           <chr> <dbl>
 1 José Luis Martínez Izquierdo    H        18
 2 Rosa Díaz Díaz                  M        32
 3 Javier García Sánchez           H        24
 4 Carmen López Pinzón             M        35
 5 Marisa López Collado            M        46
 6 Antonio Ruiz Cruz               H        68
 7 Antonio Fernández Ocaña         H        51
 8 Pilar Martín González           M        22
 9 Pedro Gálvez Tenorio            H        35
10 Santiago Reillo Manzano         H        46
11 Macarena Álvarez Luna           M        53
12 José María de la Guía Sanz      H        58
13 Miguel Angel Cuadrado Gutiérrez H        27
14 Carolina Rubio Moreno           M        20
\end{verbatim}

\end{tcolorbox}

\begin{enumerate}
\def\labelenumi{\alph{enumi}.}
\setcounter{enumi}{4}
\tightlist
\item
  Anonimizar los datos eliminando la columna \texttt{nombre}.
\end{enumerate}

\begin{tcolorbox}[enhanced jigsaw, title=\textcolor{quarto-callout-tip-color}{\faLightbulb}\hspace{0.5em}{Solución}, coltitle=black, opacitybacktitle=0.6, rightrule=.15mm, colback=white, bottomtitle=1mm, breakable, leftrule=.75mm, opacityback=0, colbacktitle=quarto-callout-tip-color!10!white, left=2mm, colframe=quarto-callout-tip-color-frame, toptitle=1mm, titlerule=0mm, arc=.35mm, bottomrule=.15mm, toprule=.15mm]

\begin{Shaded}
\begin{Highlighting}[]
\NormalTok{df }\SpecialCharTok{\%\textgreater{}\%}
    \FunctionTok{select}\NormalTok{(}\SpecialCharTok{{-}}\NormalTok{nombre)}
\end{Highlighting}
\end{Shaded}

\begin{verbatim}
# A tibble: 14 x 7
    edad sexo   peso altura colesterol   imc Obesidad 
   <dbl> <chr> <dbl>  <dbl>      <dbl> <dbl> <fct>    
 1    18 H        85   1.79        182    27 Sobrepeso
 2    32 M        65   1.73        232    22 Saludable
 3    24 H        NA   1.81        191    NA <NA>     
 4    35 M        65   1.7         200    22 Saludable
 5    46 M        51   1.58        148    20 Saludable
 6    68 H        66   1.74        249    22 Saludable
 7    51 H        62   1.72        276    21 Saludable
 8    22 M        60   1.66         NA    22 Saludable
 9    35 H        90   1.94        241    24 Saludable
10    46 H        75   1.85        280    22 Saludable
11    53 M        55   1.62        262    21 Saludable
12    58 H        78   1.87        198    22 Saludable
13    27 H       109   1.98        210    28 Sobrepeso
14    20 M        61   1.77        194    19 Saludable
\end{verbatim}

\end{tcolorbox}

\begin{enumerate}
\def\labelenumi{\alph{enumi}.}
\setcounter{enumi}{5}
\tightlist
\item
  Reordenar las columnas poniendo la columna \texttt{sexo} antes que la
  columna \texttt{edad}.
\end{enumerate}

\begin{tcolorbox}[enhanced jigsaw, title=\textcolor{quarto-callout-tip-color}{\faLightbulb}\hspace{0.5em}{Solución}, coltitle=black, opacitybacktitle=0.6, rightrule=.15mm, colback=white, bottomtitle=1mm, breakable, leftrule=.75mm, opacityback=0, colbacktitle=quarto-callout-tip-color!10!white, left=2mm, colframe=quarto-callout-tip-color-frame, toptitle=1mm, titlerule=0mm, arc=.35mm, bottomrule=.15mm, toprule=.15mm]

\begin{Shaded}
\begin{Highlighting}[]
\NormalTok{df }\SpecialCharTok{\%\textgreater{}\%}
    \FunctionTok{select}\NormalTok{(nombre, sexo, edad, }\FunctionTok{everything}\NormalTok{())}
\end{Highlighting}
\end{Shaded}

\begin{verbatim}
# A tibble: 14 x 8
   nombre                     sexo   edad  peso altura colesterol   imc Obesidad
   <chr>                      <chr> <dbl> <dbl>  <dbl>      <dbl> <dbl> <fct>   
 1 José Luis Martínez Izquie~ H        18    85   1.79        182    27 Sobrepe~
 2 Rosa Díaz Díaz             M        32    65   1.73        232    22 Saludab~
 3 Javier García Sánchez      H        24    NA   1.81        191    NA <NA>    
 4 Carmen López Pinzón        M        35    65   1.7         200    22 Saludab~
 5 Marisa López Collado       M        46    51   1.58        148    20 Saludab~
 6 Antonio Ruiz Cruz          H        68    66   1.74        249    22 Saludab~
 7 Antonio Fernández Ocaña    H        51    62   1.72        276    21 Saludab~
 8 Pilar Martín González      M        22    60   1.66         NA    22 Saludab~
 9 Pedro Gálvez Tenorio       H        35    90   1.94        241    24 Saludab~
10 Santiago Reillo Manzano    H        46    75   1.85        280    22 Saludab~
11 Macarena Álvarez Luna      M        53    55   1.62        262    21 Saludab~
12 José María de la Guía Sanz H        58    78   1.87        198    22 Saludab~
13 Miguel Angel Cuadrado Gut~ H        27   109   1.98        210    28 Sobrepe~
14 Carolina Rubio Moreno      M        20    61   1.77        194    19 Saludab~
\end{verbatim}

\end{tcolorbox}

\begin{enumerate}
\def\labelenumi{\alph{enumi}.}
\setcounter{enumi}{6}
\tightlist
\item
  Filtrar el data frame para quedarse con las mujeres.
\end{enumerate}

\begin{tcolorbox}[enhanced jigsaw, title=\textcolor{quarto-callout-tip-color}{\faLightbulb}\hspace{0.5em}{Solución}, coltitle=black, opacitybacktitle=0.6, rightrule=.15mm, colback=white, bottomtitle=1mm, breakable, leftrule=.75mm, opacityback=0, colbacktitle=quarto-callout-tip-color!10!white, left=2mm, colframe=quarto-callout-tip-color-frame, toptitle=1mm, titlerule=0mm, arc=.35mm, bottomrule=.15mm, toprule=.15mm]

\begin{Shaded}
\begin{Highlighting}[]
\NormalTok{df }\SpecialCharTok{\%\textgreater{}\%}
    \FunctionTok{filter}\NormalTok{(sexo }\SpecialCharTok{==} \StringTok{"M"}\NormalTok{)}
\end{Highlighting}
\end{Shaded}

\begin{verbatim}
# A tibble: 6 x 8
  nombre                 edad sexo   peso altura colesterol   imc Obesidad 
  <chr>                 <dbl> <chr> <dbl>  <dbl>      <dbl> <dbl> <fct>    
1 Rosa Díaz Díaz           32 M        65   1.73        232    22 Saludable
2 Carmen López Pinzón      35 M        65   1.7         200    22 Saludable
3 Marisa López Collado     46 M        51   1.58        148    20 Saludable
4 Pilar Martín González    22 M        60   1.66         NA    22 Saludable
5 Macarena Álvarez Luna    53 M        55   1.62        262    21 Saludable
6 Carolina Rubio Moreno    20 M        61   1.77        194    19 Saludable
\end{verbatim}

\end{tcolorbox}

\begin{enumerate}
\def\labelenumi{\alph{enumi}.}
\setcounter{enumi}{7}
\tightlist
\item
  Filtrar el data frame para quedarse con los hombres mayores de 30
  años.
\end{enumerate}

\begin{tcolorbox}[enhanced jigsaw, title=\textcolor{quarto-callout-tip-color}{\faLightbulb}\hspace{0.5em}{Solución}, coltitle=black, opacitybacktitle=0.6, rightrule=.15mm, colback=white, bottomtitle=1mm, breakable, leftrule=.75mm, opacityback=0, colbacktitle=quarto-callout-tip-color!10!white, left=2mm, colframe=quarto-callout-tip-color-frame, toptitle=1mm, titlerule=0mm, arc=.35mm, bottomrule=.15mm, toprule=.15mm]

\begin{Shaded}
\begin{Highlighting}[]
\NormalTok{df }\SpecialCharTok{\%\textgreater{}\%}
    \FunctionTok{filter}\NormalTok{( sexo }\SpecialCharTok{==} \StringTok{"H"} \SpecialCharTok{\&}\NormalTok{ edad }\SpecialCharTok{\textgreater{}} \DecValTok{30}\NormalTok{)}
\end{Highlighting}
\end{Shaded}

\begin{verbatim}
# A tibble: 5 x 8
  nombre                      edad sexo   peso altura colesterol   imc Obesidad 
  <chr>                      <dbl> <chr> <dbl>  <dbl>      <dbl> <dbl> <fct>    
1 Antonio Ruiz Cruz             68 H        66   1.74        249    22 Saludable
2 Antonio Fernández Ocaña       51 H        62   1.72        276    21 Saludable
3 Pedro Gálvez Tenorio          35 H        90   1.94        241    24 Saludable
4 Santiago Reillo Manzano       46 H        75   1.85        280    22 Saludable
5 José María de la Guía Sanz    58 H        78   1.87        198    22 Saludable
\end{verbatim}

\end{tcolorbox}

\begin{enumerate}
\def\labelenumi{\roman{enumi}.}
\tightlist
\item
  Filtrar el data frame para eliminar las filas con datos perdidos en la
  columna \texttt{colesterol}.
\end{enumerate}

\begin{tcolorbox}[enhanced jigsaw, title=\textcolor{quarto-callout-tip-color}{\faLightbulb}\hspace{0.5em}{Solución}, coltitle=black, opacitybacktitle=0.6, rightrule=.15mm, colback=white, bottomtitle=1mm, breakable, leftrule=.75mm, opacityback=0, colbacktitle=quarto-callout-tip-color!10!white, left=2mm, colframe=quarto-callout-tip-color-frame, toptitle=1mm, titlerule=0mm, arc=.35mm, bottomrule=.15mm, toprule=.15mm]

\begin{Shaded}
\begin{Highlighting}[]
\NormalTok{df }\SpecialCharTok{\%\textgreater{}\%}
    \FunctionTok{filter}\NormalTok{(}\SpecialCharTok{!}\FunctionTok{is.na}\NormalTok{(colesterol))}
\end{Highlighting}
\end{Shaded}

\begin{verbatim}
# A tibble: 13 x 8
   nombre                      edad sexo   peso altura colesterol   imc Obesidad
   <chr>                      <dbl> <chr> <dbl>  <dbl>      <dbl> <dbl> <fct>   
 1 José Luis Martínez Izquie~    18 H        85   1.79        182    27 Sobrepe~
 2 Rosa Díaz Díaz                32 M        65   1.73        232    22 Saludab~
 3 Javier García Sánchez         24 H        NA   1.81        191    NA <NA>    
 4 Carmen López Pinzón           35 M        65   1.7         200    22 Saludab~
 5 Marisa López Collado          46 M        51   1.58        148    20 Saludab~
 6 Antonio Ruiz Cruz             68 H        66   1.74        249    22 Saludab~
 7 Antonio Fernández Ocaña       51 H        62   1.72        276    21 Saludab~
 8 Pedro Gálvez Tenorio          35 H        90   1.94        241    24 Saludab~
 9 Santiago Reillo Manzano       46 H        75   1.85        280    22 Saludab~
10 Macarena Álvarez Luna         53 M        55   1.62        262    21 Saludab~
11 José María de la Guía Sanz    58 H        78   1.87        198    22 Saludab~
12 Miguel Angel Cuadrado Gut~    27 H       109   1.98        210    28 Sobrepe~
13 Carolina Rubio Moreno         20 M        61   1.77        194    19 Saludab~
\end{verbatim}

\end{tcolorbox}

\begin{enumerate}
\def\labelenumi{\alph{enumi}.}
\setcounter{enumi}{9}
\tightlist
\item
  Ordenar el data frame según la columna \texttt{nombre}.
\end{enumerate}

\begin{tcolorbox}[enhanced jigsaw, title=\textcolor{quarto-callout-tip-color}{\faLightbulb}\hspace{0.5em}{Solución}, coltitle=black, opacitybacktitle=0.6, rightrule=.15mm, colback=white, bottomtitle=1mm, breakable, leftrule=.75mm, opacityback=0, colbacktitle=quarto-callout-tip-color!10!white, left=2mm, colframe=quarto-callout-tip-color-frame, toptitle=1mm, titlerule=0mm, arc=.35mm, bottomrule=.15mm, toprule=.15mm]

\begin{Shaded}
\begin{Highlighting}[]
\NormalTok{df }\SpecialCharTok{\%\textgreater{}\%}
    \FunctionTok{arrange}\NormalTok{(nombre)}
\end{Highlighting}
\end{Shaded}

\begin{verbatim}
# A tibble: 14 x 8
   nombre                      edad sexo   peso altura colesterol   imc Obesidad
   <chr>                      <dbl> <chr> <dbl>  <dbl>      <dbl> <dbl> <fct>   
 1 Antonio Fernández Ocaña       51 H        62   1.72        276    21 Saludab~
 2 Antonio Ruiz Cruz             68 H        66   1.74        249    22 Saludab~
 3 Carmen López Pinzón           35 M        65   1.7         200    22 Saludab~
 4 Carolina Rubio Moreno         20 M        61   1.77        194    19 Saludab~
 5 Javier García Sánchez         24 H        NA   1.81        191    NA <NA>    
 6 José Luis Martínez Izquie~    18 H        85   1.79        182    27 Sobrepe~
 7 José María de la Guía Sanz    58 H        78   1.87        198    22 Saludab~
 8 Macarena Álvarez Luna         53 M        55   1.62        262    21 Saludab~
 9 Marisa López Collado          46 M        51   1.58        148    20 Saludab~
10 Miguel Angel Cuadrado Gut~    27 H       109   1.98        210    28 Sobrepe~
11 Pedro Gálvez Tenorio          35 H        90   1.94        241    24 Saludab~
12 Pilar Martín González         22 M        60   1.66         NA    22 Saludab~
13 Rosa Díaz Díaz                32 M        65   1.73        232    22 Saludab~
14 Santiago Reillo Manzano       46 H        75   1.85        280    22 Saludab~
\end{verbatim}

\end{tcolorbox}

\begin{enumerate}
\def\labelenumi{\alph{enumi}.}
\setcounter{enumi}{10}
\tightlist
\item
  Ordenar el data frame ascendentemente por la columna \texttt{sexo} y
  descendentemente por la columna \texttt{edad}.
\end{enumerate}

\begin{tcolorbox}[enhanced jigsaw, title=\textcolor{quarto-callout-tip-color}{\faLightbulb}\hspace{0.5em}{Solución}, coltitle=black, opacitybacktitle=0.6, rightrule=.15mm, colback=white, bottomtitle=1mm, breakable, leftrule=.75mm, opacityback=0, colbacktitle=quarto-callout-tip-color!10!white, left=2mm, colframe=quarto-callout-tip-color-frame, toptitle=1mm, titlerule=0mm, arc=.35mm, bottomrule=.15mm, toprule=.15mm]

\begin{Shaded}
\begin{Highlighting}[]
\NormalTok{df }\SpecialCharTok{\%\textgreater{}\%}
    \FunctionTok{arrange}\NormalTok{(sexo, }\FunctionTok{desc}\NormalTok{(edad))}
\end{Highlighting}
\end{Shaded}

\begin{verbatim}
# A tibble: 14 x 8
   nombre                      edad sexo   peso altura colesterol   imc Obesidad
   <chr>                      <dbl> <chr> <dbl>  <dbl>      <dbl> <dbl> <fct>   
 1 Antonio Ruiz Cruz             68 H        66   1.74        249    22 Saludab~
 2 José María de la Guía Sanz    58 H        78   1.87        198    22 Saludab~
 3 Antonio Fernández Ocaña       51 H        62   1.72        276    21 Saludab~
 4 Santiago Reillo Manzano       46 H        75   1.85        280    22 Saludab~
 5 Pedro Gálvez Tenorio          35 H        90   1.94        241    24 Saludab~
 6 Miguel Angel Cuadrado Gut~    27 H       109   1.98        210    28 Sobrepe~
 7 Javier García Sánchez         24 H        NA   1.81        191    NA <NA>    
 8 José Luis Martínez Izquie~    18 H        85   1.79        182    27 Sobrepe~
 9 Macarena Álvarez Luna         53 M        55   1.62        262    21 Saludab~
10 Marisa López Collado          46 M        51   1.58        148    20 Saludab~
11 Carmen López Pinzón           35 M        65   1.7         200    22 Saludab~
12 Rosa Díaz Díaz                32 M        65   1.73        232    22 Saludab~
13 Pilar Martín González         22 M        60   1.66         NA    22 Saludab~
14 Carolina Rubio Moreno         20 M        61   1.77        194    19 Saludab~
\end{verbatim}

\end{tcolorbox}

\end{exercise}

\hypertarget{ejercicios-propuestos}{%
\section{Ejercicios Propuestos}\label{ejercicios-propuestos}}

\leavevmode\vadjust pre{\hypertarget{exr-preprocesaimento-5}{}}%
\begin{exercise}[]\label{exr-preprocesaimento-5}

La siguiente tabla recoge las notas de los alumnos de un curso con dos
asignaturas.

\begin{longtable}[]{@{}lccc@{}}
\toprule()
Alumno & Sexo & Física & Química \\
\midrule()
\endhead
Carlos & H & 6.7 & 8.1 \\
María & M & 7.2 & 9.5 \\
Carmen & M & 5.5 & 5 \\
Pedro & H & & 4.5 \\
Luis & H & 3.5 & 5 \\
Sara & M & 6.2 & 4 \\
\bottomrule()
\end{longtable}

\begin{enumerate}
\def\labelenumi{\alph{enumi}.}
\tightlist
\item
  Definir cuatro vectores con el nombre, el sexo y las notas de Física y
  Química.
\end{enumerate}

\begin{tcolorbox}[enhanced jigsaw, title=\textcolor{quarto-callout-tip-color}{\faLightbulb}\hspace{0.5em}{Solución}, coltitle=black, opacitybacktitle=0.6, rightrule=.15mm, colback=white, bottomtitle=1mm, breakable, leftrule=.75mm, opacityback=0, colbacktitle=quarto-callout-tip-color!10!white, left=2mm, colframe=quarto-callout-tip-color-frame, toptitle=1mm, titlerule=0mm, arc=.35mm, bottomrule=.15mm, toprule=.15mm]

\begin{Shaded}
\begin{Highlighting}[]
\NormalTok{nombre }\OtherTok{\textless{}{-}} \FunctionTok{c}\NormalTok{(}\StringTok{"Carlos"}\NormalTok{, }\StringTok{"María"}\NormalTok{, }\StringTok{"Carmen"}\NormalTok{, }\StringTok{"Pedro"}\NormalTok{, }\StringTok{"Luis"}\NormalTok{, }\StringTok{"Sara"}\NormalTok{)}
\NormalTok{sexo }\OtherTok{\textless{}{-}} \FunctionTok{c}\NormalTok{(}\StringTok{"H"}\NormalTok{, }\StringTok{"M"}\NormalTok{, }\StringTok{"M"}\NormalTok{, }\StringTok{"H"}\NormalTok{, }\StringTok{"H"}\NormalTok{, }\StringTok{"M"}\NormalTok{)}
\NormalTok{fisica }\OtherTok{\textless{}{-}} \FunctionTok{c}\NormalTok{(}\FloatTok{6.7}\NormalTok{, }\FloatTok{7.2}\NormalTok{, }\FloatTok{5.5}\NormalTok{, }\ConstantTok{NA}\NormalTok{, }\FloatTok{3.5}\NormalTok{, }\FloatTok{6.2}\NormalTok{)}
\NormalTok{quimica }\OtherTok{\textless{}{-}} \FunctionTok{c}\NormalTok{(}\FloatTok{8.1}\NormalTok{, }\FloatTok{9.5}\NormalTok{, }\DecValTok{5}\NormalTok{, }\FloatTok{4.5}\NormalTok{, }\DecValTok{5}\NormalTok{, }\DecValTok{4}\NormalTok{)}
\end{Highlighting}
\end{Shaded}

\end{tcolorbox}

\begin{enumerate}
\def\labelenumi{\alph{enumi}.}
\setcounter{enumi}{1}
\tightlist
\item
  Convertir el sexo en un factor y mostrar sus niveles.
\end{enumerate}

\begin{tcolorbox}[enhanced jigsaw, title=\textcolor{quarto-callout-tip-color}{\faLightbulb}\hspace{0.5em}{Solución}, coltitle=black, opacitybacktitle=0.6, rightrule=.15mm, colback=white, bottomtitle=1mm, breakable, leftrule=.75mm, opacityback=0, colbacktitle=quarto-callout-tip-color!10!white, left=2mm, colframe=quarto-callout-tip-color-frame, toptitle=1mm, titlerule=0mm, arc=.35mm, bottomrule=.15mm, toprule=.15mm]

\begin{Shaded}
\begin{Highlighting}[]
\NormalTok{sexo }\OtherTok{\textless{}{-}} \FunctionTok{factor}\NormalTok{(sexo)}
\FunctionTok{levels}\NormalTok{(sexo)}
\end{Highlighting}
\end{Shaded}

\begin{verbatim}
[1] "H" "M"
\end{verbatim}

\end{tcolorbox}

\begin{enumerate}
\def\labelenumi{\alph{enumi}.}
\setcounter{enumi}{2}
\tightlist
\item
  Crear un data frame con el nombre, sexo y las notas de Física y
  Química.
\end{enumerate}

\begin{tcolorbox}[enhanced jigsaw, title=\textcolor{quarto-callout-tip-color}{\faLightbulb}\hspace{0.5em}{Solución}, coltitle=black, opacitybacktitle=0.6, rightrule=.15mm, colback=white, bottomtitle=1mm, breakable, leftrule=.75mm, opacityback=0, colbacktitle=quarto-callout-tip-color!10!white, left=2mm, colframe=quarto-callout-tip-color-frame, toptitle=1mm, titlerule=0mm, arc=.35mm, bottomrule=.15mm, toprule=.15mm]

\begin{Shaded}
\begin{Highlighting}[]
\NormalTok{df }\OtherTok{\textless{}{-}} \FunctionTok{data.frame}\NormalTok{(nombre, sexo, fisica, quimica)}
\NormalTok{df}
\end{Highlighting}
\end{Shaded}

\begin{verbatim}
  nombre sexo fisica quimica
1 Carlos    H    6.7     8.1
2  María    M    7.2     9.5
3 Carmen    M    5.5     5.0
4  Pedro    H     NA     4.5
5   Luis    H    3.5     5.0
6   Sara    M    6.2     4.0
\end{verbatim}

\end{tcolorbox}

\begin{enumerate}
\def\labelenumi{\alph{enumi}.}
\setcounter{enumi}{3}
\tightlist
\item
  Crear una nueva columna con la nota media de Física y Química.
\end{enumerate}

\begin{tcolorbox}[enhanced jigsaw, title=\textcolor{quarto-callout-tip-color}{\faLightbulb}\hspace{0.5em}{Solución}, coltitle=black, opacitybacktitle=0.6, rightrule=.15mm, colback=white, bottomtitle=1mm, breakable, leftrule=.75mm, opacityback=0, colbacktitle=quarto-callout-tip-color!10!white, left=2mm, colframe=quarto-callout-tip-color-frame, toptitle=1mm, titlerule=0mm, arc=.35mm, bottomrule=.15mm, toprule=.15mm]

\begin{Shaded}
\begin{Highlighting}[]
\NormalTok{df}\SpecialCharTok{$}\NormalTok{media }\OtherTok{\textless{}{-}}\NormalTok{ (df}\SpecialCharTok{$}\NormalTok{fisica }\SpecialCharTok{+}\NormalTok{ df}\SpecialCharTok{$}\NormalTok{quimica) }\SpecialCharTok{/} \DecValTok{2}
\NormalTok{df}
\end{Highlighting}
\end{Shaded}

\begin{verbatim}
  nombre sexo fisica quimica media
1 Carlos    H    6.7     8.1  7.40
2  María    M    7.2     9.5  8.35
3 Carmen    M    5.5     5.0  5.25
4  Pedro    H     NA     4.5    NA
5   Luis    H    3.5     5.0  4.25
6   Sara    M    6.2     4.0  5.10
\end{verbatim}

\end{tcolorbox}

\begin{enumerate}
\def\labelenumi{\alph{enumi}.}
\setcounter{enumi}{4}
\tightlist
\item
  Crear una nueva columna booleana \texttt{aprobado} que tenga el valor
  \texttt{TRUE} si la media es mayor o igual que 5 y \texttt{FALSE} en
  caso contrario.
\end{enumerate}

\begin{tcolorbox}[enhanced jigsaw, title=\textcolor{quarto-callout-tip-color}{\faLightbulb}\hspace{0.5em}{Solución}, coltitle=black, opacitybacktitle=0.6, rightrule=.15mm, colback=white, bottomtitle=1mm, breakable, leftrule=.75mm, opacityback=0, colbacktitle=quarto-callout-tip-color!10!white, left=2mm, colframe=quarto-callout-tip-color-frame, toptitle=1mm, titlerule=0mm, arc=.35mm, bottomrule=.15mm, toprule=.15mm]

\begin{Shaded}
\begin{Highlighting}[]
\NormalTok{df}\SpecialCharTok{$}\NormalTok{aprobado }\OtherTok{\textless{}{-}}\NormalTok{ df}\SpecialCharTok{$}\NormalTok{media }\SpecialCharTok{\textgreater{}=} \DecValTok{5}
\NormalTok{df}
\end{Highlighting}
\end{Shaded}

\begin{verbatim}
  nombre sexo fisica quimica media aprobado
1 Carlos    H    6.7     8.1  7.40     TRUE
2  María    M    7.2     9.5  8.35     TRUE
3 Carmen    M    5.5     5.0  5.25     TRUE
4  Pedro    H     NA     4.5    NA       NA
5   Luis    H    3.5     5.0  4.25    FALSE
6   Sara    M    6.2     4.0  5.10     TRUE
\end{verbatim}

\end{tcolorbox}

\begin{enumerate}
\def\labelenumi{\alph{enumi}.}
\setcounter{enumi}{5}
\tightlist
\item
  Filtrar el data frame para quedarse con el nombre y la media de las
  mujeres que han aprobado.
\end{enumerate}

\begin{tcolorbox}[enhanced jigsaw, title=\textcolor{quarto-callout-tip-color}{\faLightbulb}\hspace{0.5em}{Solución}, coltitle=black, opacitybacktitle=0.6, rightrule=.15mm, colback=white, bottomtitle=1mm, breakable, leftrule=.75mm, opacityback=0, colbacktitle=quarto-callout-tip-color!10!white, left=2mm, colframe=quarto-callout-tip-color-frame, toptitle=1mm, titlerule=0mm, arc=.35mm, bottomrule=.15mm, toprule=.15mm]

\begin{Shaded}
\begin{Highlighting}[]
\NormalTok{df[df}\SpecialCharTok{$}\NormalTok{sexo }\SpecialCharTok{==} \StringTok{"M"} \SpecialCharTok{\&}\NormalTok{ df}\SpecialCharTok{$}\NormalTok{media }\SpecialCharTok{\textgreater{}=} \DecValTok{5}\NormalTok{, }\FunctionTok{c}\NormalTok{(}\StringTok{"nombre"}\NormalTok{, }\StringTok{"media"}\NormalTok{)]}
\end{Highlighting}
\end{Shaded}

\begin{verbatim}
  nombre media
2  María  8.35
3 Carmen  5.25
6   Sara  5.10
\end{verbatim}

\end{tcolorbox}

\end{exercise}

\leavevmode\vadjust pre{\hypertarget{exr-preprocesamiento-5}{}}%
\begin{exercise}[]\label{exr-preprocesamiento-5}

Se ha diseñado un ensayo clínico leatorizado, doble-ciego y controlado
con placebo, para estudiar el efecto de dos alternativas terapéuticas en
el control de la hipertensión arterial. Se han reclutado 100 pacientes
hipertensos y estos han sido distribuidos aleatoriamente en tres grupos
de tratamiento. A uno de los grupos (control) se le administró un
placebo, a otro grupo se le administró un inhibidor de la enzima
conversora de la angiotensina (IECA) y al otro un tratamiento combinado
de un diurético y un Antagonista del Calcio. Las variables respuesta
final fueron las presiones arteriales sistólica y diastólica.

Los datos con las claves de aleatorización han sido introducidos en una
base de datos que reside en la central de aleatorización, mientras que
los datos clínicos han sido archivados en dos archivos distintos, uno
para cada uno de los dos centros participantes en el estudio.

Las variables almacenadas en estos archivos clínicos son las siguientes:

\begin{itemize}
\tightlist
\item
  CLAVE: Clave de aleatorización
\item
  NOMBRE: Iniciales del paciente
\item
  F\_NACIM: Fecha de Nacimiento
\item
  F\_INCLUS: Fecha de inclusión
\item
  SEXO: Sexo (0: Hombre 1: Mujer)
\item
  ALTURA: Altura en cm.
\item
  PESO: Peso en Kg.
\item
  PAD\_INI: Presión diastólica basal (inicial)
\item
  PAD\_FIN: Presión diastólica final
\item
  PAS\_INI: Presión sistólica basal (inicial)
\item
  PAS\_FIN: Presión sistólica final
\end{itemize}

El archivo de claves de aleatorización contiene sólo dos variables.

\begin{itemize}
\tightlist
\item
  CLAVE: Clave de aleatorización
\item
  FARMACO: Fármaco administrado (0: Placebo, 1: IECA, 2:Ca Antagonista +
  diurético)
\end{itemize}

\begin{enumerate}
\def\labelenumi{\alph{enumi}.}
\tightlist
\item
  Crear un data frame con los datos de los pacientes del hospital A
  (fichero de Excel
  \href{datos/hipertension/datos-hospital-a.xls}{datos-hospital-a.xls}).
\end{enumerate}

\begin{tcolorbox}[enhanced jigsaw, title=\textcolor{quarto-callout-tip-color}{\faLightbulb}\hspace{0.5em}{Solución}, coltitle=black, opacitybacktitle=0.6, rightrule=.15mm, colback=white, bottomtitle=1mm, breakable, leftrule=.75mm, opacityback=0, colbacktitle=quarto-callout-tip-color!10!white, left=2mm, colframe=quarto-callout-tip-color-frame, toptitle=1mm, titlerule=0mm, arc=.35mm, bottomrule=.15mm, toprule=.15mm]

\begin{Shaded}
\begin{Highlighting}[]
\FunctionTok{library}\NormalTok{(readxl)}
\CommentTok{\#dfa \textless{}{-} read\_excel()}
\end{Highlighting}
\end{Shaded}

\end{tcolorbox}

\begin{enumerate}
\def\labelenumi{\alph{enumi}.}
\setcounter{enumi}{1}
\item
  Crear un data frame con los datos de los pacientes del hosptial B
  (fichero csv
  \href{datos/hipertension/datos-hospital-b.csv}{datos-hospital-b.csv}).
\item
  Fusionar los datos de los dos hospitales en un nuevo data frame.
\item
  Crear un data frame con los datos de las claves de aleatorización
  (fichero csv
  \href{datos/hipertension/claves-aleatorizacion.csv}{claves-aleatorizacion.csv})
\item
  Fusionar el data frame con los datos clínicos y el data frame con
  claves de aleatorización en un nuevo data frame.
\item
  Convertir en un factor el sexo.
\item
  Crear una nueva columna para la evolución de la presión arterial
  diastólica restando las columnas \texttt{PAS\_FIN} y
  \texttt{PAS\_FIN}.
\end{enumerate}

\end{exercise}



\end{document}
